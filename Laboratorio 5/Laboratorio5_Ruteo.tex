\documentclass[a4paper,10pt]{article}

\usepackage[margin=2cm, tmargin=1.5in, headheight=40px]{geometry}
\usepackage[sfdefault, medium, light]{roboto}
%\usepackage[nomap]{FiraMono}
\usepackage[T1]{fontenc}
\usepackage[utf8]{inputenc}
\usepackage[spanish]{babel}
\usepackage{graphicx}
\usepackage{fancyhdr}
\usepackage{minted}
\usepackage[svgnames]{xcolor}
\setminted{style=default,bgcolor=Gainsboro}
\usepackage[allcolors={DarkSlateBlue}, colorlinks]{hyperref}

\fancyhf{}
\fancyhead[L]{\includegraphics[height=35px]{../figuras/logoort.png}}
\fancyhead[R]{Facultad de Ingeniería \\ Cátedra de Redes y Sistemas de Comunicación}
\fancyfoot[R]{Página \thepage}
\fancyfoot[L]{Redes de Datos -- Laboratorio 3}
\renewcommand{\footrulewidth}{0.2pt}
\renewcommand{\headrulewidth}{0.2pt}

\pagestyle{fancy}


\title{\bf Redes de Datos\\Laboratorio 5 -- Capa de Red\\ Ruteo dinámico}
\author{\bf Universidad ORT Uruguay}
\date{\bf Curso 2025}

\begin{document}

\maketitle
\thispagestyle{fancy}

En este laboratorio trabajaremos a nivel de Capa de Red, utilizando equipos Cisco disponibles en el Laboratorio. Los objetivos son continuar a partir de lo trabajado en el Laboratorio 4, implementando \textbf{ruteo dinámico}. En particular:

\begin{itemize}
    \item Configuraremos una topología dada, con sus interfaces y direcciones.
    \item Implementaremos ruteo dinámico para ver cómo se completan las tablas de ruteo de la topología.
    \item Interconectaremos las topologías para armar una red simulando la red de un ISP grande, o la interconexión de dos ISP.
\end{itemize}

Nuevamente, disponemos en el laboratorio de Routers pre-cableados en topología de anillo, utilizando enlaces \textbf{seriales} punto a punto. Los routers disponen además de interfaces \textbf{Ethernet} (FastEthernet o GigabitEthernet, según el modelo). Sin embargo, los routers inicialmente tienen todas sus interfaces apagadas y sin configurar.

La topología base del laboratorio es la siguiente:

\bigskip

\begin{center}
    \includegraphics[width=0.85\textwidth]{../figuras/topologia_lab.png}
\end{center}

Durante la práctica, cada grupo se encargará de configurar \emph{un único router} y su LAN asociada. Sin embargo, debe coordinar con sus compañeros para que la asignación de direcciones y rutas sea \emph{coherente}.

\vfill

\section{Configuración de interfaces}

\begin{enumerate}
    \item Retome la asignación de direcciones del Laboratorio 4 para su topología, utilizando para ello el rango de IPs \texttt{192.168.100.0/23} o bien \texttt{192.168.200.0/23}, de acuerdo al anillo que se encuentre.
    \item Configure cada una de las interfaces de su router, coordinando con los demás routers de su anillo.
    \item Verifique que:
    \begin{enumerate}
        \item Su tabla de ruteo (\mintinline{batch}{RTRX> show ip route}) quedó adecuadamente configurada \emph{solo con las interfaces directamente conectadas}.
        \item Verifique conectividad con sus routers vecinos (\mintinline{batch}{RTRX> ping <IP>}).
    \end{enumerate}

Puede usar la siguiente tabla para registrar la asignación:

    \begin{center}
        \bigskip
        \begin{tabular}{|p{1cm}|p{3cm}|p{3cm}|p{3cm}|}
            \hline
            \textbf{Red} & \textbf{Dirección de red} & \textbf{Máscara} & \textbf{IP Router} \\ \hline
            \textbf{LAN A} & & & \\ \hline
            \textbf{LAN B} & & & \\ \hline
            \textbf{LAN C} & & & \\ \hline
            \textbf{LAN D} & & & \\ \hline
        \end{tabular}
    \end{center}

    \bigskip
    \begin{center}
        \begin{tabular}{|p{2cm}|p{3cm}|p{3cm}|p{3cm}|p{3cm}|}
            \hline
            \textbf{Enlace} & \textbf{Dirección de red} & \textbf{Máscara} & \textbf{IP Router A} & \textbf{IP Router B} \\ \hline
            \textbf{RTR1--RTR2} & & & & \\ \hline
            \textbf{RTR2--RTR3} & & & & \\ \hline
            \textbf{RTR3--RTR4} & & & & \\ \hline
            \textbf{RTR4--RTR1} & & & & \\ \hline
        \end{tabular}
    \bigskip
    \end{center}

\end{enumerate}

\textbf{Nota:} Recuerde que para acceder al router puede utilizar el equipo remotizador de consola disponible en el laboratorio, mediante:
\begin{minted}{batch}
> telnet 172.20.20.24
> Login: grupoX
> Password: ort
\end{minted}
donde X es el no. de router.

\section{Conexión de máquinas de usuario}

\begin{enumerate}
    \item Implemente la red local conectando un Switch a la interfaz Ethernet del router. Verifique que ahora la misma queda activa.
    \item Conecte su máquina del laboratorio a dicha red mediante un patch-cord adecuado, utilizando la tarjeta de red secundaria de la máquina.
    \item Configure la dirección IP, máscara y puerta de enlace predeterminada de su máquina para que quede conectada a la LAN adecuada.
    \item Chequee conectividad con el Router mediante \texttt{ping}. ¿Puede llegar más allá?
\end{enumerate}

\section{Ruteo dinámico intra-red: OSPF}\label{sec:ospf1}

\begin{enumerate}
    \item Configure OSPF en todos los routers. Asegúrese de que el protocolo quede activo
en todas las interfaces. Utilice el número de área \texttt{2} si se encuentra en la red \texttt{192.168.100.0/22} y \texttt{2} si está en \texttt{192.168.200.0/22}.

    \item Verifique que el Router logra aprender todos los rangos de su topología. ¿Qué distancia administativa asigna? ¿Qué significa esto? ¿Cuál es la métrica de la ruta?
    
    \item Verifique que ahora puede alcanzar toda la red de su anillo desde cualquier PC.
    
    \item Pruebe cambiar ahora el parámetro \texttt{bandwidth} de una interfaz para ver el efecto en la métrica.
    
    \item Coordine con sus compañeros para desconectar una interfaz. ¿Cuánto demora el protocolo en reaccionar al cambio y recalcular las rutas? ¿Por qué?
\end{enumerate}

\section{Ruteo dinámico intra-red: OSPF y áreas}

Para esta parte, conectaremos ambos anillos entre sí mediante un enlace ``backbone'' (\texttt{area 0}).

\begin{enumerate}
    \item Coordine con su equipo cuál será el router de salida de su anillo. Conéctelo con el correspondiente al otro anillo utilizando una interfaz Ethernet.
    \item Configure las interfaces para usar el rango de direcciones \texttt{192.168.1.0/30} en dicho enlace.
    \item Habilite OSPF para la nueva red (solo en el router de salida), utilizando el área \texttt{area 0}.
    \item Verifique la nueva tabla de rutas del Router. ¿Logra llegar a toda la red? ¿Cómo se indican las nuevas rutas?
    \item Habilite la sumarización por áreas en los routers del backbone. ¿Qué cambia?
    \item ¿Cuál es la ventaja de utilizar áreas en OSPF?
\end{enumerate}

\section{Ruteo dinámico externo: BGP}

En esta parte, simularemos que cada anillo corresponde a un ISP diferente, y que intercambian información de ruteo mediante BGP, que es un protocolo de pasarela exterior. Antes de comenzar esta parte, \textbf{deshabilite OSPF en el enlace de backbone} de manera de volver a la situación de la sección \ref{sec:ospf1}.

\begin{enumerate}
    \item Habilite el intercambio BGP entre los routers que conectan ambos anillos.
    \item Una vez establecido, ¿logra ver toda la red desde los routers? ¿Por qué?
    \item Habilite la redistribución de rutas entre ambos protocolos. Verifique que ahora sí es posible acceder a toda la red.
\end{enumerate}
\appendix

\section{Comandos Cisco}

\textbf{Comandos básicos:}

\begin{itemize}
    \item Ingresar al modo administrador:
\mintinline{batch}{RTRX> enable}
    \item Mostrar la configuración actual: \mintinline{batch}{RTRX# show running-config}
    \item Mostrar las interfaces (resumen): \mintinline{batch}{RTRX# show ip interface brief}
    \item Mostrar la tabla de rutas: \mintinline{batch}{RTRX# show ip route}
    \item Entrar al modo configuración: \mintinline{batch}{RTRX# configure terminal}
    \item Entrar a la conf. de interfaz (ej: FastEthernet 0/0): \mintinline{batch}{RTRX(config)# interface FastEthernet 0/0}
    \item Asignar IP a una interfaz: \mintinline{batch}{RTRX(config-if)# ip address <IP> <mascara>}
    
    donde la IP y la máscara van en formato decimal separadas por puntos.
    \item Encender una interfaz: \mintinline{batch}{RTRX(config-if)# no shutdown}
    \item Asignar IP de broadcast: \mintinline{batch}{RTRX(config-if)# ip broadcast-address <IP>}
    \item Crear una ruta estática: \mintinline{batch}{RTRX(config)# ip route <dir. Red> <máscara> <nextHop>}
    \item Salir de un submenú: \mintinline{batch}{RTRX(config)# exit}

\end{itemize}

\textbf{Comandos OSPF}

\begin{itemize}
    \item Habilitar OSPF \mintinline{batch}{RTRX#(config)> router ospf <num_proceso>}
    \item Agregar redes a la topología OSPF: 
    
    \mintinline{batch}{RTRX(config-router)# network <IP_red> <wildcard> area <num_area>}
    
    Se deben agregar una por una todas las redes en que se desee utilizar. \texttt{wildcard} es el complemento a 1 de la máscara, utilizado por razones históricas. Ejemplo: máscara \texttt{255.255.255.192} $\mapsto$ wildcard \texttt{0.0.0.63}.

    \item Habilitar sumarización por área: \mintinline{batch}{RTRX(config-router)# area <num_area> range <IP_red> <mask>}
    
    Esto debe hacerse sólo en los routers intra-área.

    \item Mostrar información de OSPF: \mintinline{batch}{RTRX# show ip ospf database}
    
    \item Redistribuir rutas de otro protocolo: \mintinline{batch}{RTRX(config-router)# redistribute <protocol> <id>}

\end{itemize}

\textbf{Comandos BGP}

\begin{itemize}
    \item Habilitar BGP \mintinline{batch}{RTRX#(config)> router bgp <num_AS>}
    \item Definir identificador BGP \mintinline{batch}{RTRX#(config-router)> bgp router-id <IP>}
    \item Establecer una vecindad BGP: 
    
    \mintinline{batch}{RTRX(config-router)#neighbor <IP_remota> remote-as <num_AS_remoto>}
    
    \item Resumir tablas de rutas: \mintinline{batch}{RTRX(config-router)# auto-summary}
    
    \item Mostrar información de BGP: \mintinline{batch}{RTRX# show ip bgp}

    \item Redistribuir rutas de otro protocolo:
    
    \mintinline{batch}{RTRX(config-router)# redistribute <protocol> <id> metric <default_metric>}
    
\end{itemize}


\end{document}